% Poster originally adapted from James Lloyd's template
% https://github.com/jamesrobertlloyd/cbl-tikz-poster

%%%%%%%%%%%%%%%%%%%%%%%%%%%%%%%%%%%%%%%%%%%
%
% Document class
%
%%%%%%%%%%%%%%%%%%%%%%%%%%%%%%%%%%%%%%%%%%%

%\documentclass[landscape,a0b,final,a4resizeable]{a0poster}
%\documentclass[portrait,a0b,final,a4resizeable]{a0poster}
\documentclass[portrait,a0b,final]{a0poster}

%%%%%%%%%%%%%%%%%%%%%%%%%%%%%%%%%%%%%%%%%%%
%
% Basic packages
%
%%%%%%%%%%%%%%%%%%%%%%%%%%%%%%%%%%%%%%%%%%%

\usepackage[usenames,dvipsnames]{xcolor}
\usepackage{multicol}
\usepackage{color}
\usepackage{shadow}
\usepackage{morefloats}
\usepackage[pdftex]{graphicx}
\usepackage{rotating}
\usepackage{amsmath,amsthm,amssymb,amsfonts}
\usepackage{mathtools}
\usepackage{array}
\usepackage{nth}
\usepackage{booktabs}
\usepackage{dsfont}
\usepackage{parskip}
\usepackage{enumitem}
\usepackage[round,comma]{natbib}

% Bibliography without title
\makeatletter
\renewcommand\@biblabel[1]{}
\renewenvironment{thebibliography}[1]
{%\section*{\refname}%
\@mkboth{\MakeUppercase\refname}{\MakeUppercase\refname}%
\list{}%
{\leftmargin0pt
\@openbib@code
\usecounter{enumiv}}%
\sloppy
\clubpenalty4000
\@clubpenalty \clubpenalty
\widowpenalty4000%
\sfcode`\.\@m}
{\def\@noitemerr
{\@latex@warning{Empty `thebibliography' environment}}%
\endlist}
\makeatother

\renewcommand{\bibsection}{}

% Caption for use outside floats
\newcommand{\mycaption}[1]{\parbox{\linewidth}{\large \vspace{10pt} #1 \vspace{10pt}}}

% Subfigures, etc.
\usepackage{subfigure}

%\usepackage{pagegrid}
\usepackage[OT1]{fontenc}
\usepackage{eucal}

% My maths macros
\usepackage{/Users/kelly/Workspace/Utilities/luke-macros}

\newcommand{\oemph}[1]{\textcolor{Orange}{#1}}
\newcommand{\bemph}[1]{\textcolor{RoyalBlue}{#1}}

%%%%%%%%%%%%%%%%%%%%%%%%%%%%%%%%%%%%%%%%%%%
%
% TIKZ packages and common definitions
%
%%%%%%%%%%%%%%%%%%%%%%%%%%%%%%%%%%%%%%%%%%%

\usepackage{tikz}
\usetikzlibrary{shapes,shapes.geometric,arrows,chains,matrix,positioning,scopes,calc}

\tikzset{
    nodeStyleLeft/.style={text width=0.3\textwidth, anchor=north, align=justify, inner sep=0},
    nodeStyleRight/.style={text width=0.65\textwidth, anchor=north west, align=justify, inner sep=0}
}

\graphicspath{
    {/Users/kelly/Documents/Papers/22-Coupling/aoas}
}

%%%%%%%%%%%%%%%%%%%%%%%%%%%%%%%%%%%%%%%%%%%
%
% Some standard colours
%
%%%%%%%%%%%%%%%%%%%%%%%%%%%%%%%%%%%%%%%%%%%

\definecolor{dauphinedarkblue}{rgb}{0.165,0.24,0.44}

%%%%%%%%%%%%%%%%%%%%%%%%%%%%%%%%%%%%%%%%%%%
%
% Some look and feel definitions
%
%%%%%%%%%%%%%%%%%%%%%%%%%%%%%%%%%%%%%%%%%%%

\setlength{\columnsep}{0.05\textwidth}
\setlength{\columnseprule}{0.00025\textwidth}
% \setlength{\parindent}{1cm}
% \setlength{\parskip}{1cm}

%%%%%%%%%%%%%%%%%%%%%%%%%%%%%%%%%%%%%%%%%%%
%
% \mysection - replacement for \section*
%
%%%%%%%%%%%%%%%%%%%%%%%%%%%%%%%%%%%%%%%%%%%

\tikzset{mysection/.style={
    rectangle,
    draw=none,
    shade,
    outer color=dauphinedarkblue,
    inner color=dauphinedarkblue,
    text width=0.97\columnwidth,
    text centered,
    text=white,
    rounded corners=20pt,
    minimum height=3cm
}}

\newcommand{\mysection}[1]
{
	\begin{center}
		\begin{tikzpicture}
			\node[mysection] {\bfseries\LARGE#1};
		\end{tikzpicture}
	\end{center}
}

%%%%%%%%%%%%%%%%%%%%%%%%%%%%%%%%%%%%%%%%%%%%
%%
%% \myalign - replacement for {align*}
%%
%%%%%%%%%%%%%%%%%%%%%%%%%%%%%%%%%%%%%%%%%%%%
%
%\tikzset{myalign/.style={draw, rectangle,
%	text width=\columnwidth,
%	text centered,
%	text=black}}
%
%\newcommand{\myalign}[1]
%{
%	\begin{center}
%		\begin{tikzpicture}
%			\node[myalign] {\Large $ \begin{aligned} #1 \end{aligned} $};
%		\end{tikzpicture}
%	\end{center}
%}

%%%%%%%%%%%%%%%%%%%%%%%%%%%%%%%%%%%%%%%%%%%
%
% Set the font
%
%%%%%%%%%%%%%%%%%%%%%%%%%%%%%%%%%%%%%%%%%%%

\renewcommand{\familydefault}{cmss}
\sffamily

%%%%%%%%%%%%%%%%%%%%%%%%%%%%%%%%%%%%%%%%%%%
%
% Poster environment
%
%%%%%%%%%%%%%%%%%%%%%%%%%%%%%%%%%%%%%%%%%%%

\newenvironment{poster}
{
    \begin{center}
    	\hspace{-2in}
    	\begin{minipage}[c]{0.96\textwidth}
}
{
    	\end{minipage}
    \end{center}
}

%%%%%%%%%%%%%%%%%%%%%%%%%%%%%%%%%%%%%%%%%%%
%
% The document environment starts here
%
%%%%%%%%%%%%%%%%%%%%%%%%%%%%%%%%%%%%%%%%%%%

\begin{document}

%%%%%%%%%%%%%%%%%%%%%%%%%%%%%%%%%%%%%%%%%%%
%
% Begin the poster environment - centres things etc
%
%%%%%%%%%%%%%%%%%%%%%%%%%%%%%%%%%%%%%%%%%%%

\begin{poster}

% \vspace{1\baselineskip}

%%%%%%%%%%%%%%%%%%%%%%%%%%%%%%%%%%%%%%%%%%%
%
% Draw the header as a TIKZ picture
%
%%%%%%%%%%%%%%%%%%%%%%%%%%%%%%%%%%%%%%%%%%%

\begin{center}
	\begin{tikzpicture}
		\begin{scope}
			\node[anchor=west, inner sep=0, text width=0.3\textwidth] (logo) at (0, 0) {
				\includegraphics[width=\textwidth]{/Users/kelly/Workspace/Utilities/Ceremade/Seul/Ceremade-Blue-Tutelle-CMJN}
			};
			\node[anchor=west, inner sep=0, text width=0.65\textwidth, text centered, font=\Huge] (Title) at ($ (logo.east) + (0.05\textwidth, 0) $) {
				{\Huge \textbf{Lagged couplings for phylogenetic inference}}\\ {\huge Diagnosing convergence of MCMC on trees}\\
				{\huge Luke J. Kelly, Robin J. Ryder and Grégoire Clarté}
			};
		\end{scope}
	\end{tikzpicture}
\end{center}

%%%%%%%%%%%%%%%%%%%%%%%%%%%%%%%%%%%%%%%%%%%
%
% Spacing between title and main body
%
%%%%%%%%%%%%%%%%%%%%%%%%%%%%%%%%%%%%%%%%%%%

\vspace{1\baselineskip}

%%%%%%%%%%%%%%%%%%%%%%%%%%%%%%%%%%%%%%%%%%%
%
% Main body
%
%%%%%%%%%%%%%%%%%%%%%%%%%%%%%%%%%%%%%%%%%%%

\LARGE

\begin{tikzpicture}
\begin{scope}

\node (n0) [nodeStyleLeft] at (0, 0) {
    \mysection{Summary}
    Existing methods do not diagnose MCMC convergence jointly across all components of a phylogenetic model
    \\[0.5\baselineskip plus 2pt]
    We couple Markov chains $ (X_s)_s $ and $ (Y_s)_s $ with stationary distribution $ \pi $ on \bemph{trees, model parameters and latent variables}
    \\[0.5\baselineskip plus 2pt]
    Chains coupled at lag $ l $ meet exactly at random finite time $ \tau^{(l)} $ and remain together
    \\[0.5\baselineskip plus 2pt]
    Estimate upper bound on convergence
    \[
        d_{\mathrm{TV}}(\pi_s, \pi) \leq \EE\left[0 \vee \ceil*{\frac{\tau^{(l)} - l - s}{l}}\right]
    \]
};

\node (n1) [nodeStyleLeft] at ($ (n0.south) + (0, -0.025\textwidth) $)
{
    \mysection{Phylogenetics}

    Phylogenetics is the problem of inferring the ancestral history of a set of species
    \begin{itemize}[itemsep=6pt,topsep=8pt]
        \item Common traits indicate shared ancestry
        \item Phylogeny is typically a tree with data recorded at leaves
    \end{itemize}
    \vspace{0.5\baselineskip plus 2pt}

    \begin{center}
        \includegraphics[width=\textwidth]{figs/phytree-c}
        \mycaption{
            A branching process \tikz{\node[draw,shape=circle,minimum size=16pt,fill=white] {};} on sets of traits is the phylogeny of the observed taxa \tikz{\node[draw,shape=circle,minimum size=16pt,fill=black] {};}, catastrophes \tikz{\node[draw,circle,minimum size=16pt,fill=Goldenrod] {};} represent bursts of evolutionary activity, trait sets diversify according to the Stochastic Dollo model
            \tikz{\node[draw,shape=diamond,minimum size=20pt,fill=Turquoise] {};} \tikz{\node[draw,star,star points=7,star point ratio=0.8,minimum size=16pt,fill=Mulberry] {};} \tikz{\node[draw,regular polygon,minimum size=22pt,fill=NavyBlue] {};}
        }
    \end{center}
};

%%%%%%%%%%%%%%%%%%%%%%%%%%%%%%%%%%%%%%%%%%%

\node (n2) [nodeStyleLeft] at ($ (n1.south) + (0, -0.025\textwidth) $)
{
    \mysection{Inference problem}

    Difficult statistical problem
    \begin{itemize}[itemsep=6pt,topsep=8pt]
        \item Infer tree topology, branch lengths, rate parameters and catastrophe locations
        \item Intractable likelihood, often multimodal
        \item Many constraints and dependencies
    \end{itemize}
    \vspace{0.5\baselineskip plus 2pt}

    MCMC for Bayesian phylogenetic inference
    \begin{itemize}[itemsep=6pt,topsep=8pt]
        \item Theoretical properties largely unknown
        \item Existing approaches compare many low-dimensional summaries so lack power to quantify MCMC behaviour in practice
        \item Difficult to separate modelling and fitting errors in inference
    \end{itemize}
};

%%%%%%%%%%%%%%%%%%%%%%%%%%%%%%%%%%%%%%%%%%%

\node (n3) [nodeStyleLeft] at ($ (n2.south) + (0, -0.025\textwidth) $)
{
    \mysection{References}
    \nocite{jacob20,biswas19,kelly21}
    \normalsize
    \bibliographystyle{unsrtnat}
    \bibliography{references}
};

%%%%%%%%%%%%%%%%%%%%%%%%%%%%%%%%%%%%%%%%%%

\node (n5) [nodeStyleRight] at ($ (n0.north east) + (0.05\textwidth, 0) $)
{
    \mysection{Coupling MCMC for phylogenetic models}

    In our mixture of proposal kernels, subtree prune-and-regraft (SPR) moves are the primary tool for exploring tree space

    \begin{center}
        \includegraphics[width=\linewidth]{archive/spr}
        \mycaption{\centering SPR proposal moves the \textcolor{YellowGreen}{subtree with root $ i $} to the \textcolor{BlueGreen}{branch leading into $ j $} while respecting ancestry constraints}
    \end{center}

    \oemph{Implicitly coupling moves}: relabel nodes to identify similar tree components in current states $ (X, Y) $ and sample from a maximal coupling at each step of move to form proposals $ (X', Y') $

    \begin{multicols}{2}
        \begin{center}
            \includegraphics[width=\linewidth]{figs/spr-x-j5-c}
            \mycaption{\centering SPR $ X \rightarrow X' $: move subtree \textcolor{YellowGreen}{$ i^{(X)} = 1 $} to branch \textcolor{BlueGreen}{$ j^{(X)} = 5 $}}
        \end{center}
        \begin{center}
            \includegraphics[width=\linewidth]{figs/spr-y-j5-c}
            \mycaption{\centering SPR $ Y \rightarrow Y' $: move subtree \textcolor{YellowGreen}{$ i^{(Y)} = 1 $} to branch \textcolor{BlueGreen}{$ j^{(Y)} = 5 $}}
        \end{center}
    \end{multicols}
};

%%%%%%%%%%%%%%%%%%%%%%%%%%%%%%%%%%%%%%%%%%%%

\node (n6) [nodeStyleRight] at ($ (n5.south west) + (0, -0.025\textwidth) $)
{
    \mysection{Eastern Polynesian languages}

    Fitting the Stochastic Dollo model to lexical trait data in 11 languages

    \begin{multicols}{2}
        \begin{center}
            \includegraphics[width=\linewidth, trim = 0cm 0cm 0cm 1.25cm, clip]{analysis-figs/20210807-polynesian/tv}
            \mycaption{Estimated TV bounds on tree and model parameters, 100 pairs of chains at each lag  $ l $}
        \end{center}
        \begin{center}
            \includegraphics[width=\linewidth, trim = 0cm 0cm 0cm 1.25cm, clip]{analysis-figs/20210807-polynesian/asdsf}
            \mycaption{Average Standard Deviation of Split Frequencies (ASDSF) on sliding windows across 100 chains}
        \end{center}
    \end{multicols}

    \bemph{Estimated TV bound diagnoses convergence across all components of the model and earlier than ASDSF or other methods}
};

\node (n7) [nodeStyleRight] at ($ (n6.south west) + (0, -0.025\textwidth) $)
{
    \mysection{Larger trees}

    Coupling trees with 100 and 200 leaves, synthetic data with death rate $ \mu $

    \vspace{6pt}
    \begin{center}
        \includegraphics[width=\linewidth, trim = 0cm 0cm 0cm 1.25cm, clip]{analysis-figs/20220419-large/tau-eccdf_axes-free_x}
    \end{center}
};

\node (n8) [nodeStyleRight] at ($ (n7.south west) + (0, -0.025\textwidth) $)
{
    \mysection{Future work}
    \begin{multicols}{2}
        \begin{itemize}[itemsep=6pt,topsep=8pt]
            \item Maximally coupling moves on trees
            \item Coupling trees with thousands of leaves
            \item Identify mixing or modelling issues from chains which fail to meet
        \end{itemize}
    \end{multicols}
};

%%%%%%%%%%%%%%%%%%%%%%%%%%%%%%%%%%%%%%%%%%%
%
% Decorative lines.
%
%%%%%%%%%%%%%%%%%%%%%%%%%%%%%%%%%%%%%%%%%%%

\node (l1) at ($ (n0.north east) + (0.025\linewidth, 0) $) {};
\node (l2) at ($ (n3.south east) + (0.025\linewidth, 0) $) {};
\draw[black] (l1) -- (l2);

\end{scope}
\end{tikzpicture}

\end{poster}

\end{document}
