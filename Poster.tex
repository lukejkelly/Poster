% Poster originally adapted from James Lloyd's template 
% https://github.com/jamesrobertlloyd/cbl-tikz-poster

%%%%%%%%%%%%%%%%%%%%%%%%%%%%%%%%%%%%%%%%%%%
%
% Document class
%
%%%%%%%%%%%%%%%%%%%%%%%%%%%%%%%%%%%%%%%%%%%

%\documentclass[landscape,a0b,final,a4resizeable]{a0poster}
%\documentclass[portrait,a0b,final,a4resizeable]{a0poster}
%\documentclass[portrait,a0,final]{a0poster}
\documentclass[portrait,a0b,final]{a0poster}

%%%%%%%%%%%%%%%%%%%%%%%%%%%%%%%%%%%%%%%%%%%
%
% Basic packages
%
%%%%%%%%%%%%%%%%%%%%%%%%%%%%%%%%%%%%%%%%%%%

\usepackage[usenames,dvipsnames]{xcolor}
\usepackage{multicol}
\usepackage{color}
\usepackage{shadow}
\usepackage{morefloats}
\usepackage[pdftex]{graphicx}
\usepackage{rotating}
\usepackage{amsmath, amsthm, amssymb, amsfonts}
\usepackage{array}
\usepackage{nth}
\usepackage{booktabs}
\usepackage{bbm}

% Bibliography without title.
\usepackage[square, numbers, comma]{natbib}

% Remove numbers from bibligraphy
\makeatletter
\renewcommand\@biblabel[1]{}
\renewenvironment{thebibliography}[1]
  {%\section*{\refname}%
  \@mkboth{\MakeUppercase\refname}{\MakeUppercase\refname}%
  \list{}%
  {\leftmargin0pt
  \@openbib@code
  \usecounter{enumiv}}%
  \sloppy
  \clubpenalty4000
  \@clubpenalty \clubpenalty
  \widowpenalty4000%
  \sfcode`\.\@m}
  {\def\@noitemerr
   {\@latex@warning{Empty `thebibliography' environment}}%
  \endlist}
\makeatother

\renewcommand{\bibsection}{}

% Caption for use outside floats.
\newcommand{\myCaption}[1]{\parbox{\linewidth}{\large \vspace{10pt} #1 \vspace{10pt}}}

% Subfigures, etc.
\usepackage{subfigure}

%\usepackage{pagegrid}
\usepackage[OT1]{fontenc}
\usepackage{eucal}

% My maths macros.
\usepackage{mathsMacros}

% Orange emphasis.
\newcommand{\oemph}[1]{\textcolor{orange}{#1}}
\newcommand{\bemph}[1]{\textcolor{RoyalBlue}{#1}}
\definecolor{myturquoise}{rgb}{0 0.41 0.41}
\definecolor{myblue}{rgb}{0.25, 0.75, 1.00}
\definecolor{mygreen}{rgb}{0.75, 1.00, 0.00}
\definecolor{myred}{rgb}{1.00, 0.50, 0.50}

%%%%%%%%%%%%%%%%%%%%%%%%%%%%%%%%%%%%%%%%%%%
%
% TIKZ packages and common definitions
%
%%%%%%%%%%%%%%%%%%%%%%%%%%%%%%%%%%%%%%%%%%%

\usepackage{picins}
\usepackage{tikz}
\usetikzlibrary{shapes, shapes.geometric, arrows, chains, matrix, positioning, scopes, calc}
\tikzstyle{mybox} = [draw=white, rectangle]

\graphicspath{{PosterFigures/}{ThesisFigures/}{PaperFigures/}{Branding/}}

%%%%%%%%%%%%%%%%%%%%%%%%%%%%%%%%%%%%%%%%%%%
%
% Some standard colours
%
%%%%%%%%%%%%%%%%%%%%%%%%%%%%%%%%%%%%%%%%%%%

\definecolor{oxdarkblue}{RGB}{4, 30, 66}

%%%%%%%%%%%%%%%%%%%%%%%%%%%%%%%%%%%%%%%%%%%
%
% Some look and feel definitions
%
%%%%%%%%%%%%%%%%%%%%%%%%%%%%%%%%%%%%%%%%%%%

\setlength{\columnsep}{0.05\textwidth}
\setlength{\columnseprule}{0.00025\textwidth}
\setlength{\parindent}{1cm}
\setlength{\parskip}{1cm}

%%%%%%%%%%%%%%%%%%%%%%%%%%%%%%%%%%%%%%%%%%%
%
% \mysection - replacement for \section*
%
%%%%%%%%%%%%%%%%%%%%%%%%%%%%%%%%%%%%%%%%%%%

\tikzstyle{mysection} = [rectangle,
	draw=none,
	shade,
	outer color=oxdarkblue,
	inner color=oxdarkblue,
	text width=0.97\columnwidth,
	text centered,
	text=white,
	rounded corners=20pt,
	minimum height=3cm]%0.11\columnwidth]

\newcommand{\mysection}[1]
{
	\begin{center}
		\begin{tikzpicture}
			\node[mysection] {\sffamily\bfseries\LARGE#1};
		\end{tikzpicture}
	\end{center}
}

%%%%%%%%%%%%%%%%%%%%%%%%%%%%%%%%%%%%%%%%%%%%
%%
%% \myalign - replacement for {align*}
%%
%%%%%%%%%%%%%%%%%%%%%%%%%%%%%%%%%%%%%%%%%%%%
%
%\tikzstyle{myalign} = [draw, rectangle,
%	text width=\columnwidth,
%	text centered,
%	text=black]
%
%\newcommand{\myalign}[1]
%{
%	\begin{center}
%		\begin{tikzpicture}
%			\node[myalign] {\Large $ \begin{aligned} #1 \end{aligned} $};
%		\end{tikzpicture}
%	\end{center}
%}

%%%%%%%%%%%%%%%%%%%%%%%%%%%%%%%%%%%%%%%%%%%
%
% Set the font
%
%%%%%%%%%%%%%%%%%%%%%%%%%%%%%%%%%%%%%%%%%%%

\renewcommand{\familydefault}{cmss}
\sffamily

%%%%%%%%%%%%%%%%%%%%%%%%%%%%%%%%%%%%%%%%%%%
%
% Poster environment
%
%%%%%%%%%%%%%%%%%%%%%%%%%%%%%%%%%%%%%%%%%%%

\newenvironment{poster}
{
	\begin{center}
		\hspace{-2in}
		\begin{minipage}[c]{0.96\textwidth}
		}
		{
		\end{minipage}
	\end{center}
}

%%%%%%%%%%%%%%%%%%%%%%%%%%%%%%%%%%%%%%%%%%%
%
% The document environment starts here
%
%%%%%%%%%%%%%%%%%%%%%%%%%%%%%%%%%%%%%%%%%%%

\begin{document}

%%%%%%%%%%%%%%%%%%%%%%%%%%%%%%%%%%%%%%%%%%%
%
% Begin the poster environment - centres things etc
%
%%%%%%%%%%%%%%%%%%%%%%%%%%%%%%%%%%%%%%%%%%%

\begin{poster}

% \vspace{1\baselineskip}

%%%%%%%%%%%%%%%%%%%%%%%%%%%%%%%%%%%%%%%%%%%
%
% Draw the header as a TIKZ picture
%
%%%%%%%%%%%%%%%%%%%%%%%%%%%%%%%%%%%%%%%%%%%

\begin{center}
	\begin{tikzpicture}
		\begin{scope}
			\node[anchor=west, inner sep=0, text width=0.3\textwidth] (logo) at (0, 0) {
				\includegraphics[width=\textwidth]{Branding/Dept_Stats_logo_horizontal_CMYK.eps}
			};
			\node[anchor=west, inner sep=0, text width=0.65\textwidth, text centered, font=\Huge] (Title) at ($ (logo.east) + (0.05\textwidth, 0) $) {
				{\sffamily\Huge \textbf{Accelerated inference in a complex phylogenetic model}\\ {\huge Exact inference with massive systems of ODEs}}\\
				{\sffamily\huge Luke Kelly and Geoff Nicholls}%\\
			};
		\end{scope}
	\end{tikzpicture}
\end{center}

%%%%%%%%%%%%%%%%%%%%%%%%%%%%%%%%%%%%%%%%%%%
%
% Spacing between title and main body
%
%%%%%%%%%%%%%%%%%%%%%%%%%%%%%%%%%%%%%%%%%%%

\vspace{1\baselineskip}

%%%%%%%%%%%%%%%%%%%%%%%%%%%%%%%%%%%%%%%%%%%
%
% Main body
%
%%%%%%%%%%%%%%%%%%%%%%%%%%%%%%%%%%%%%%%%%%%

\LARGE

\begin{tikzpicture}
\begin{scope}

\node (n1) [text width=0.3\textwidth, align = justify, anchor = north, inner sep = 0] at (0, 0)
{
\mysection{Motivation}

Lateral trait transfer is a form of evolutionary activity whereby traits pass through non-ancestral relationships between contemporary species.

\setlength{\parindent}{1cm}

Although tree-like, the histories of transferred traits conflict with the overall phylogeny so models based solely on ancestral inheritance are \oemph{misspecified} here.

\begin{center}
\includegraphics[width=\textwidth]{PhytreeDataVH.pdf}
% \myCaption{A phylogenetic tree represents the ancestry of the observed taxa. A taxon comprises a set of \emph{homologous} traits, where traits derived from a common ancestor by speciation or lateral transfer events are homologous.}
\end{center}
};

%%%%%%%%%%%%%%%%%%%%%%%%%%%%%%%%%%%%%%%%%%%

\node (n2) [text width=0.3\textwidth, align = justify, anchor = north, inner sep = 0] at ($ (n1.south) + (0, -0.025\textwidth) $)
{
\mysection{Process}
\definecolor{turquoise}{rgb}{0 0.41 0.41}

A \textcolor{myblue}{branching process} \tikz{\draw[draw, thick, fill = myblue] circle [radius = 12pt];} on sets of traits is the phylogeny of the \textcolor{mygreen}{observed taxa} \tikz{\draw[draw, fill = mygreen] circle [radius = 12pt];}.
\begin{itemize}
\item Species \textcolor{myturquoise!50}{evolve new} \tikz{\node[rectangle, draw, rotate=45, minimum width=20pt, minimum height=20pt, inner sep=0pt, text width=0pt, fill=myturquoise!50] {};} traits at rate $ \lambda $.
\item Trait instances \textcolor{Mulberry!50}{die} {\tikz{\node[draw, star, star points = 7, star point ratio = 0.8, minimum width=20pt, minimum height=20pt, inner sep=0pt, text width=0pt, fill=magenta!50] {};}} independently at rate $ \mu $ and \textcolor{Orange}{transfer} \tikz{\draw[draw, fill = Orange] circle [radius = 12pt];} to other species at per capita rate $ \beta $.
\end{itemize}

\begin{center}
\includegraphics[width=\textwidth, trim = 1.1cm 0cm 0cm 0cm, clip]{PhytreePatternEvolve/PhytreePatternEvolve1236.pdf}
\myCaption{\large A phylogenetic tree and \textcolor{red}{trait history} drawn from our process with snapshots of the corresponding pattern process. \textcolor{Goldenrod}{Catastrophe nodes} \tikz{\draw[draw, fill = Goldenrod] circle [radius = 10pt];} represent spikes in activity.}
\end{center}

A trait $ h $ displays a pattern $ \pht \in \cPt = \{0, 1\}^{\Lt} \setminus \{\mathbf{0}\} $ of presence or absence across $ \Lt $ species at time $ t $. Traits are exchangeable so we model the \textcolor{SpringGreen!75}{terminal pattern} frequencies $ \bN(T) = (\Np(T))_{\bp \in \cP^{(T)}} $.
};

%%%%%%%%%%%%%%%%%%%%%%%%%%%%%%%%%%%%%%%%%%%

\node (n3) [text width=0.3\textwidth, align = justify, anchor = north, inner sep = 0] at ($ (n2.south) + (0, -0.025\textwidth) $)
{
\mysection{References}
\nocite{jennings71,kelly16,kelly17}
\large
\bibliographystyle{unsrtnat}
\bibliography{references}
};

%%%%%%%%%%%%%%%%%%%%%%%%%%%%%%%%%%%%%%%%%%

\node (n4) [text width=0.3\textwidth, align = justify, anchor = north, inner sep = 0] at ($ (n3.south) + (0, -0.025\textwidth) $)
{
\mysection{Acknowledgements}
\Large
St John's College and the EPSRC.
};

%%%%%%%%%%%%%%%%%%%%%%%%%%%%%%%%%%%%%%%%%%%

\node (n5a) [text width=0.65\textwidth, align = justify, anchor = north west, inner sep = 0] at ($ (n1.north east) + (0.05\textwidth, 0) $)
{
\mysection{Inference}

\begin{multicols}{2}

After \oemph{integrating out} the birth rate $ \lambda $ and \textcolor{red}{unobserved trait histories}, $ \bN(T) $ is multinomial with unnormalised weights $ \bx(T) = (\xp(T))_{\bp \in \cP^{(T)}} $, the solution of a \oemph{sequence of initial value problems across the tree}.

\setlength{\parindent}{1cm}

The ODEs have dimension $ \cO(2^{\oemph{\Lt}}) $ so exact inference with an ODE solver quickly becomes intractable as $ \Lt $ increases.

\begin{center}
\includegraphics[width=\columnwidth, trim = 0.45cm 0cm 0cm 0cm, clip]{PatternMeanTransfer.pdf}
\end{center}

\end{multicols}
\bemph{We want to perform MCMC so develop a fast method for computing the parameters $ \bx(T) $.}
};

%%%%%%%%%%%%%%%%%%%%%%%%%%%%%%%%%%%%%%%%%%%%

\node (n5b) [text width=0.65\textwidth, align = justify, anchor = north west, inner sep = 0] at ($ (n5a.south west) + (0, -0.025\textwidth) $)
{
\mysection{Symmetries and equivalence classes}

\begin{center}
\includegraphics[width=0.28\linewidth]{Cube3EC.pdf}
\includegraphics[width=0.23\linewidth]{EquivClasses001.pdf}
\includegraphics[width=0.23\linewidth]{EquivClasses011.pdf}
\includegraphics[width=0.23\linewidth]{EquivClasses111.pdf}
\end{center}

For $ \by^{(1)}, \dotsc, \by^{(\Lt)} $ and $ \bz $ solving IVPs on $ \cO({\Lt}^2) $ equivalence classes, we have
\[
\xp(t + \Delta) = \sum_{\bq \in \cPt} y^{s(\bq)}_{s(\bp), d(\bp, \bq)}(\Delta) \xqt + z_{\bp}(\Delta), \qquad
\textcolor{black!50}{
\left\{ \begin{array}{ll}
  \text{Hamming weight } & s \\
  \text{Hamming distance } & d
\end{array} \right\}}.
\]
The computational cost of this exact approach is $ \cO(2^{\oemph{2 \Lt}}) $ as we form $ \exp(\bAt \Delta) $ explicitly.

\begin{multicols}{2}

Sparse estimator $ \bG(\Delta) $ of $ \exp(\bAt \Delta) $,
\[
  G_{\bp, \bq}(\Delta) = \left\{
  \begin{array}{ll}
    y^{s(\bq)}_{s(\bq), d(\bp, \bq)}(\Delta), \quad & d(\bp, \bq) \leq 1, \\
    0, \quad & \text{otherwise},
  \end{array} \right.
\]
and construct $ \bx^{(0)}, \bx^{(1)}, \ldots \rightarrow \bx $ by
\[
  \bx^{(k)}(t + \Delta) = {\bG(\Delta 2^{-k})}^{2^k} \bx(t) + \bZ(\Delta).
\]
\oemph{Linear convergence}: $ \bx - \bx^{(k)} = \cO(2^{-k}) \bx $.
\end{multicols}
};

%%%%%%%%%%%%%%%%%%%%%%%%%%%%%%%%%%%%%%%%%%%

\node (n5c) [text width=0.65\textwidth, align = justify, anchor = north west, inner sep = 0] at ($ (n5b.south west) + (0, -0.025\textwidth) $)
{
\mysection{Acceleration scheme}

\definecolor{g0}{rgb}{0, 1, 0}
\definecolor{g1}{rgb}{0, 1, 0.25}
\definecolor{g2}{rgb}{0, 1, 0.5}
\definecolor{g3}{rgb}{0, 1, 0.75}
\definecolor{g4}{rgb}{0, 1, 1}

\definecolor{r0}{rgb}{1, 0, 0}
\definecolor{r1}{rgb}{1, 0, 0.5}
\definecolor{r2}{rgb}{1, 0, 1}

\definecolor{o0}{rgb}{1, 0.5, 0}

\begin{multicols}{2}
  \oemph{Jennings' transformation}, a stable, non-linear extrapolation for vector sequences, \oemph{significantly reduces the error} in our estimates with negligible computational cost.
\begin{center}
  \includegraphics[width=\linewidth, trim = 1.6cm 0cm 2cm 0.5cm, clip]{gfMaxError.pdf}
  \myCaption{$ \textcolor{g0}{\bx^{(0)}}, \textcolor{g1}{\bx^{(1)}}, \textcolor{g2}{\bx^{(2)}}, \textcolor{g3}{\bx^{(3)}}, \textcolor{g4}{\bx^{(4)}} \xrightarrow{\mathrm{Jennings}} \textcolor{r0}{\bv^{(0)}}, \textcolor{r1}{\bv^{(1)}}, \textcolor{r2}{\bv^{(2)}} \xrightarrow{\mathrm{Jennings}} \textcolor{o0}{\tilde{\bx}} $.}
\end{center}

Construct an \oemph{unbiased likelihood estimator} and run pseudo-marginal MCMC.
  \begin{center}
    \includegraphics[width=0.8\linewidth, trim = 0.05cm 0cm 0.6cm 0.25cm, clip]{likeExactvsDiff.pdf}
  \end{center}
\end{multicols}

\begin{itemize}
\item[\textcolor{orange}{\textbullet}] \bemph{Our accelerated inference scheme is exact in a MCMC sense.}
\item[\textcolor{orange}{\textbullet}] \bemph{The effective sample size per unit time is an order of magnitude higher than computing parameters with a standard ODE solver and running the Metropolis--Hastings algorithm.}
\end{itemize}
};

%%%%%%%%%%%%%%%%%%%%%%%%%%%%%%%%%%%%%%%%%%%

%\node (n6) [text width=0.3\textwidth, align = justify, anchor = north west, inner sep = 0] at ($ (n51.south west) + (0, -0.025\textwidth) $)
%{
%\mysection{Future work}
%};

%%%%%%%%%%%%%%%%%%%%%%%%%%%%%%%%%%%%%%%%%%%

% \node (n7) [text width=0.3\textwidth, align = justify, anchor = north west, inner sep = 0] at ($ (n5c.south west) + (0, -0.025\textwidth) $)
% {
% \mysection{References}
% \normalsize
% \bibliographystyle{unsrt}
% \bibliography{references}
% };

%%%%%%%%%%%%%%%%%%%%%%%%%%%%%%%%%%%%%%%%%%%

%\node (n8) [text width=0.3\textwidth, align = justify, anchor = north, inner sep = 0] at ($ (n6.south) + (0, -0.025\textwidth) $)
%{
%\mysection{}
%\vspace{5\baselineskip}
%\dots
%};

%%%%%%%%%%%%%%%%%%%%%%%%%%%%%%%%%%%%%%%%%%%

% \node (n9) [text width=0.3\textwidth, align = justify, anchor = north west, inner sep = 0] at ($ (n7.north east) + (0.05\textwidth, 0) $)
% {
% \mysection{Acknowledgements}
%
% Financial support provided by St John's College and the EPSRC.
% };

%%%%%%%%%%%%%%%%%%%%%%%%%%%%%%%%%%%%%%%%%%%
%
% Decorative lines.
%
%%%%%%%%%%%%%%%%%%%%%%%%%%%%%%%%%%%%%%%%%%%

\node (l1) at ($ (n1.north east) + (0.025\linewidth, 0) $) {};
\node (l2) at ($ (n4.south east) + (0.025\linewidth, 0) $) {};
\draw[black] (l1) -- (l2);

% \node (l3) at ($ (n7.north east) + (0.025\linewidth, 0) $) {};
% \node (l4) at ($ (l2) + (0.35\linewidth, 0) $) {};
% \draw[black] (l3) -- (l4);

\end{scope}
\end{tikzpicture}

\end{poster}

\end{document}
